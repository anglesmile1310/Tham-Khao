\documentclass[../cnd.tex]{subfiles}

Ethereum được xem như một máy trạng thái dựa trên các giao dịch. Chúng ta bắt đầu với trạng thái khởi tạo và thực hiện các giao dịch để dần biến đổi nó tới trạng thái kết thúc. Mỗi trạng thái có thể bao gồm một số thông tin như số dư tài khoản, tên tài khoản, dữ liệu liên quan đến thông tin về thế giới vật lý hay bất cứ điều gì mà có thể đại diện bởi một máy tính được chấp nhận. Vì vậy, các giao dịch phải đại diện cho đường vòng cung nối giữa hai trạng thái hợp lệ. Sự hợp lệ trong giao dịch là vô cùng quan trọng. Quá trình chuyển đổi trạng thái hợp lệ là một quá trình chuyển đổi thông qua giao dịch. Như sau: 

	\begin{equation}
	\sigma_{t+1} \equiv \Upsilon(\sigma_{t} , T)
	\end{equation}

Trong đó $\Upsilon$ là hàm chuyển đổi trạng thái trong Ethereum. Trong Ethereum,$\Upsilon$ cùng với $\sigma$ mạnh hơn đáng kể so với bất kỳ hệ thống nào đang có. $\Upsilon$ cho phép có thể tính toán tùy ý, trong khi đó $\sigma$ cho phép lưu các trạng thái tùy ý giữa các giao dịch.

Các giao dịch được sắp xếp thành các khối; các khối được nối liền với nhau bằng cách sử dụng một băm mật mã. Các khối hoạt động như một nhật ký, ghi lại một loạt các giao dịch cùng với khối trước đó và một định danh cho trạng thái cuối cùng (mặc dù không lưu trữ trạng thái cuối cùng chính nó | sẽ quá lớn). Hệ thống cũng chấm dứt chuỗi giao dịch với phần thưởng cho người đào. Sự khuyến khích này diễn ra như là một chức năng chuyển đổi trạng thái, bổ sung giá trị cho một tài khoản được chỉ định.

Nó được giải thích như sau:
\begin{align}
\sigma_{t+1} \equiv \Pi(\sigma_{t} , B) \\
B \equiv (...,(T_0,T_1,...))\\
\Pi(\sigma , B) \equiv \Omega(B, \Upsilon(\Upsilon(\sigma,T_0),T_1)...)
\end{align}

Trong đó $\Omega$ là chức năng chuyển trạng thái hoàn thiện khối(chức năng thưởng cho bên chỉ định); B là khối này, bao gồm một loạt các giao dịch giữa một số thành phần khác; và $\Pi$ là cấp của khối trong hàm chuyển trạng thái.

Đây là cơ sở của mô hình BlockChain, một mô hình tạo thành xương sống không chỉ của Ethereum, mà còn cho đến nay tất cả các hệ thống giao dịch dựa trên sự đồng thuận giữa các bên.